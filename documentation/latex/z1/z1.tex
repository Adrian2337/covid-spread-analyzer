\documentclass[11pt]{article}
\usepackage[T1]{fontenc}
\usepackage{hyperref}
\begin{document}
\title{Programowanie zespołowese\\Zadanie 1 - faza wstępna}
\maketitle
\section{Skład grupy projektowej}
\begin{enumerate}
\item Popis Piotr (kierownik)
\item Kazimierski Jakub
\item Wałejko Mateusz
\item Majcher Adrian
\end{enumerate}
\section{Temat i zakres projektu}
Tematem naszego projektu jest sytuacja epidemiologiczna panująca w Polsce. Mianowicie chcemy stworzyć aplikację webową, która pozwoli na poszerzanie wiedzy oraz analizę w zakresie rozpowszechniania się koronowirusa. Przykładowo poza podsystemami rejestracji, logowania i ogólnej obsługi użytkownika będzie prowadził statystyki na podstawie "dotychczas" zebranych danych i generował odpowiadające im wykresy ( np. ilośc zarażeń od czasu). Z biegiem czasu na wykresach będzie można zauważyć pewne charakterystyczne zachowania dla danych okresów czasowych. Aktualne ilości zakażeń będą wyświetlane na responsywnej mapce. Aplikacja będzie codziennie aktualizowała swoją bazę danych o ilości przypadków na danym obszarze w danym dniu, które zostaną pobrane z oficjalnych źródeł.  
\section{Technologie}
\begin{enumerate}
\item Python
\item Django
\item CSS
\item HTML
\item Javascript
\item Firebase
\end{enumerate}
\section{Zarządzanie projektem}
\paragraph{Metodologia:} SCRUM
\paragraph{Platforma:}
TRELLO
\section{Repozytorium}
 \href{https://github.com/sqoshi/team-programming-project}{https://github.com/sqoshi/team-programming-project}.
\end{document}